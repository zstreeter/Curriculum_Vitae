\documentclass{scrartcl}

\reversemarginpar% Move the margin to the left of the page

\newcommand{\MarginText}[1]{\marginpar{\raggedleft\itshape\small#1}} % New command defining the margin text style
\usepackage[utf8]{inputenc}
\usepackage[T1]{fontenc}
\usepackage[nochapters]{classicthesis} % Use the classicthesis style for the style of the document
\usepackage[LabelsAligned]{currvita} % Use the currvita style for the layout of the document

\renewcommand{\cvheadingfont}{\LARGE\color{BrickRed}} % Font color of your name at the top

\usepackage{hyperref} % Required for adding links	and customizing them
\hypersetup{colorlinks, breaklinks, urlcolor=BrickRed, linkcolor=BrickRed} % Set link colors

\newlength{\datebox}\settowidth{\datebox}{0.001mm } % Set the width of the date box in each block

\newcommand{\NewEntry}[3]{\noindent\hangindent=0.5em\hangafter=0 \parbox[t]{\datebox}{\small \textit{#1}} \hspace{1.5em}#2 #3  % Define a command for each new block - change spacing and font sizes here: #1 is the left margin, #2 is the italic date field and #3 is the position/employer/location field
\vspace{0.5em}} % Add some white space after each new entry

\newcommand{\Description}[1]{\hangindent=0.5em\hangafter=0\noindent\raggedright\footnotesize{#1}\par\normalsize\vspace{1em}} % Define a command for descriptions of each entry - change spacing and font sizes here

%----------------------------------------------------------------------------------------

\begin{document}

\thispagestyle{empty} % Stop the page count at the bottom of the first page

%----------------------------------------------------------------------------------------
%	NAME AND CONTACT INFORMATION SECTION
%----------------------------------------------------------------------------------------

\begin{cv}{\spacedallcaps{Zachary Streeter}}\vspace{1.5em} % Your name

\noindent\spacedlowsmallcaps{Personal Information}\vspace{0.5em} % Personal information heading

\hrule \vspace{1em}
% \hline \vspace{1em}
% \titlerule \vspace{1em}

% \NewEntry{}{\textit{Born in West Monroe, Louisiana,}}{$23$ May $1988$} % Birthplace and date

\NewEntry{email}{\href{mailto:zacharylouis42@gmail.com}{zacharylouis$42$@gmail.com}} % Email address

\NewEntry{LinkedIn}{\href{https://www.linkedin.com/in/zachary-streeter-44a323102/}{https://www.linkedin.com/in/zachary-streeter-44a323102/}}
% LinkedIn

\NewEntry{github}{\href{https://github.com/zstreeter}{https://github.com/zstreeter}} % github

%\NewEntry{website}{\href{http://www.johnsmith.com}{http://www.johnsmith.com}} % Personal website
% \NewEntry{Address}{$7000$ Gentle Oak Dr. Austin, Tx $78749$}

% \NewEntry{phone}{(M) +$1$ ($318$) $614$ $6728$}  %Phone number(s) (H) +1 (000) 111 1111\ \ $\cdotp$\ \

% \NewEntry{Familial Status}{Single, no children}

\vspace{1em} % Extra white space between the personal information section and goal

\noindent\spacedlowsmallcaps{Brief Introduction}\vspace{1em} % Goal heading, could be used for a quotation or short profile instead

\Description{My formal training is in computational physics and chemistry. 
These fields, in particular, have had limited success because of their 
massive combinatorial search spaces. This has lead to approximate techniques 
like density-functional theory (DFT) in quantum chemistry. However, these 
approximate techniques haven't yielded \textit{ab-initio} understanding 
which, in some ways, has lead to stagnation. There are two novel solution 
paths for these statistically daunting areas of study, namely quantum 
computing and AI. With this in mind I have begun my career after my Ph.D., 
in AI. With this industry experience, I will create tools that leverage 
AI for teaching science with an initial objective of teaching physics. 
As these tools mature, I hope they will lead to \textit{ab initio} 
understanding of emergent phenomena in physics and other fields like 
biology. This should keep me busy the rest of my life.\newline

If you would like, please follow the red links above to email me, link up on
LinkedIn, and/or check out my github page!}

\vspace{2em} % Goal text

%Programming------------------------------------------------------
\spacedlowsmallcaps{Technical Skills}\vspace{0.5em}

% \titlerule \vspace{1em}
\hrule \vspace{1em}

\NewEntry{Software}{}

\Description{\ \ \textsc{Compiled}\ \ $\cdotp$\ \ C(proficient), C++(proficient), Fortran(proficient), Cython(prior experience).}

\Description{\ \ \textsc{Parallel API}\ \ $\cdotp$\ \ MPI(proficient), OpenMP(proficient), Cuda(prior experience), PETSc(proficient), SLEPc(proficient).}

\Description{\ \ \textsc{Scripting}\ \ $\cdotp$\ \ Posix(prior experience), Bash(proficient), Python(proficient).}

\Description{\ \ \textsc{Build Process}\ \ $\cdotp$\ \ CMake(proficient), Make(proficient).}

\Description{\ \ \textsc{Markup}\ \ $\cdotp$\ \ {\LaTeX}(expert), Markdown(proficient), ReStructuredText(proficient).}

\Description{\ \ \textsc{Debugger}\ \ $\cdotp$\ \ gdb(proficient), lldb(proficient), TotalView(prior experience).}

\Description{\ \ \textsc{Profiler}\ \ $\cdotp$\ \ Nsight/visual profiler(prior exexperience), VTune(prior experience).}

\Description{\ \ \textsc{Scheduler}\ \ $\cdotp$\ \ SLURM(proficient).}

\NewEntry{Workflow}{}

\Description{\ \ \textsc{Editor}\ \ $\cdotp$\ \ Vim/Neovim.}

\Description{\ \ \textsc{Multiplexer}\ \ $\cdotp$\ \ Tmux.}

\Description{\ \ \textsc{Version-Control}\ \ $\cdotp$\ \ Git.}

%---------------------------------------------------------------------------------------
%RESEARCH INTEREST
%---------------------------------------------------------------------------------------

\spacedlowsmallcaps{Research Interests}\vspace{0.5em}

% \titlerule \vspace{1em}
\hrule \vspace{1em}

%Computer Science------------------------------------------------------
\NewEntry{Computer Science}{High Performance Computing}

\Description{\ \ $\cdotp$\ \ Deep Learning.}

\Description{\ \ $\cdotp$\ \ Numerical algorithms/methods.}

\Description{\ \ $\cdotp$\ \ Finite-Element Methods.}

\Description{\ \ $\cdotp$\ \ Computational geometry.}

\Description{\ \ $\cdotp$\ \ Computational physics/chemistry.}

\Description{\ \ $\cdotp$\ \ Hybrid CPU/GPU Architectures.}

\Description{\ \ $\cdotp$\ \ HPC and low level optimization.}

%Theoretical---------------------------------------------------------
\NewEntry{Theoretical}{Physics and Chemistry}

\Description{\ \ $\cdotp$\ \ Quantum Information and Computation.}

\Description{\ \ $\cdotp$\ \ Deep Learning applied to quantum physics/chemistry.}

\Description{\ \ $\cdotp$\ \ Quantum Computers applied to quantum physics/chemistry.}

\Description{\ \ $\cdotp$\ \ Nonlinear chemical reaction kinetics.}

\Description{\ \ $\cdotp$\ \ Scattering Theory.}

\Description{\ \ $\cdotp$\ \ Symplectic Mechanics.}

\Description{\ \ $\cdotp$\ \ Underlying Symmetries throughout Physics.}

%----------------------------------------------------------------------------------------
%	WORK EXPERIENCE
%----------------------------------------------------------------------------------------
\noindent\spacedlowsmallcaps{Jobs}\vspace{0.5em}

\hrule \vspace{1em}

\NewEntry{September 2021 to present}{Advanced Micro Devices}

\Description{\MarginText{AMD}Part of the Deep Learning Frameworks team. Worked on
improving novel AI models with end-to-end specifications for AMD hardware.
\begin{itemize}
  \item Collaborated with several large clients on various tasks.
  \item Major contributor to an internal project that provides dashboards and metrics on the performance of novel AI models using AMD hardware.
  \item Directed projects that significantly influenced the company's AI strategy.
    \begin{itemize}
      \item \textbf{Clusters:} Leveraged my experience at NERSC to ensure the hardware setup and software stack followed best practices for AI scaling.
      \item \textbf{Knowledge:} Equipped the team with essential frameworks for analyzing multi-node performance, including strong and weak scaling, and using roofline analysis for targeted optimizations.
    \end{itemize}
  \item Some contributions to frameworks (not including creating/maintaining ROCm ports):
    \begin{itemize}
      \item \textbf{PyTorch:} Led ROCm compatibility with Inductor/triton for the day one release of 2.0 functionality.
      \item \textbf{Triton:} Managed compatibility on our Navi lineup, overseeing the development of triton implementation to leverage WMMA instructions.
      \item \textbf{TensorFlow:} Led compatibility and performance enhancements on our Navi lineup.
      \item \textbf{XLA:} Helped team understand cost model and developed the AMD port of the Profile Guided Latency Estimator targeting multi-GPU collectives.
      \item \textbf{Jax:} Led Navi compatibility efforts.
      \item \textbf{TinyGrad:} Developed a fuzzer (fuzzyHSA) that exposes a Python API for the Kernel Fusion driver, facilitating quicker investigation of the attack surface of AMD GPUs.
    \end{itemize}
  \item Used profilers like OmniPerf and OmniTrace to identify bottlenecks and optimize kernels.
  \item Advanced understanding of AI research directions and advised on how AMD can leverage new neural-network architectures.
  \item Regarded as "Professor-in-house" and led a paper series to enhance understanding of novel techniques as the AI industry evolves.
  \item Mentor summer interns so they have an enjoyable and educational experience at AMD!
\end{itemize}
Manager: Peng \textsc{Sun}\ \ $\cdotp$\ \ \href{mailto:Peng.Sun@amd.com}{Peng.Sun@amd.com}}


%----------------------------------------------------------------------------------------
%	INTERNSHIPS and RESEARCH EXPERIENCE
%----------------------------------------------------------------------------------------

\noindent\spacedlowsmallcaps{Internships and Research Positions}\vspace{0.5em}

\hrule \vspace{1em}

\NewEntry{Summer 2016 to August 2021}{Lawrence Berkeley National Laboratory}

\Description{\MarginText{LBNL}Created fully dimensional potential energy
surfaces for H$_2$O$^{++}$ using MOLPRO and Columbus Quantum Chemistry packages.
These hypersurfaces were then used in a MPI parallelized classical trajectory simulation of
H$_2$O$^{++}$ breakup following double ionization. This work was essential to
deduce the body-frame of the water molecule at the momentum of photo-absorption
and resulted in two immediate papers while also providing a benchmark for
intense field experimentalist that will be in print shortly. Created a novel
suite of high-performance codes that calculate double-ionization cross section
for water and can be easily modified to other polyatomics.
In general, honed programming skills in C, C++, Fortran, and Python, while
becoming a learned software developer devoted to best practices, high
performance, and good
documentation.  Used NERSC
supercomputers EDISON and CORI, and also a cluster called Lawrencium, for
running large parallel batch jobs (e.g. $~40+$ physical cores with $~3000+$ processors).  Became
proficient in parallel programing using PETSC, MPI, CUDA, and OpenMP.\@ \\ Reference: Clyde W. McCurdy\ \ +$1$ ($510$) $486$ $4283$\ \ $\cdotp$\ \ \href{mailto:cwmmccurdy@lbl.gov}{cwmccurdy@lbl.gov}}

\NewEntry{Spring 2015}{Brookhaven National Laboratory, SULI internship}

\Description{\MarginText{BNL}Performed experiments with soft X-rays utilizing
the Linear Electron Accelerator Facility (LEAF) and the van de Graaff.  Prepared
samples in glove box and worked on purifying Xenon and CO.\@ This work was essential in
studying electron mobility through CO.\@ Once this work was completed, we
calculated the quasi-free electron energy resulting in a publication. Understanding the
free-electron energy in various liquids is critical in order for those
liquids to be used in scattering experiments. \\ Reference: Richard Holroyd\ \ +$1$ ($631$) $344$ $4329$\ \ $\cdotp$\ \ \href{mailto:holroydr@optonline.net}{holroydr@optonline.net}}

\NewEntry{Summer 2014}{Center for Advanced Microstructures and Devices}

\Description{\MarginText{CAMD}Became a user in order to continue research from
SRC.\@ \\ Reference: Cherice \textsc{Evans}\ \  +$1$ ($718$) $997$ $4216$\ \ $\cdotp$\ \ \href{mailto:cherice.evans@qc.cuny.edu}{cherice.evans@qc.cuny.edu}}

\NewEntry{2012--2013}{Synchrotron Radiation Center}

\Description{\MarginText{SRC}Built gas handling systems, ran leak checks for high vacuum line, wrote Igor Pro code for data analysis, and worked on calibrating the monochrometer.  Also attended lectures in relativistic electrodynamics and worked on electrodynamic problem sets.\\ Reference: Gary \textsc{Findley}\ \  +$1$ ($318$) $342$ $1835$\ \ $\cdotp$\ \ \href{mailto:findley@ulm.edu}{findley@ulm.edu}}

%------------------------------------------------

\vspace{3em} % Extra space between major sections

%---------------------------------------------------------------------------------------
%OPEN SOURCE PROJECTS
%---------------------------------------------------------------------------------------

\spacedlowsmallcaps{Open Source Projects}\vspace{0.5em}

% \titlerule \vspace{1em}
\hrule \vspace{1em}

\NewEntry{Spring 2021}{\href{https://github.com/yourusername/fuzzyHSA}{fuzzyHSA}}

\Description{\MarginText{Author and Maintainer}fuzzyHSA is a python API designed to interact with the Kernel Fusion Driver (KFD) on AMD GPUs, serving as a tool for fuzz testing the lower-level software stack. Fuzz testing involves sending unexpected or random inputs to the hardware to uncover vulnerabilities and ensure robustness. The main features of fuzzyHSA include:
  \begin{itemize}
    \item Simplified API for interacting with KFD.
    \item Enhanced debugging capabilities for kernel fusion operations.
    \item Fuzz testing to assess the robustness of AMD hardware by provoking unintended behavior and observing how gracefully the system handles these scenarios.
  \end{itemize}
  fuzzyHSA aims to provide a clear insight into the hardware's robustness by identifying how the firmware or other managing programs respond to atypical inputs. This project is currently in active development, continuously evolving to meet the needs of GPU programming and kernel fusion optimization, ensuring AMD hardware can handle unexpected inputs without compromising stability.
}

\NewEntry{Spring 2020}{\href{https://quantumgrid.readthedocs.io/en/latest/}{quantumGrid}} % github

\Description{\MarginText{Author and Maintainer}quantumGrid is a python package for solving a $1$-D
Schrödinger equation for an arbitrary potential on any interval. The heart of
this package is using a Finite Element Method with a Discrete Variable
Representation (FEM-DVR) grid to solve the time-dependent or time-independent
Schrödinger equation. This grid provides a compact supported foundation for
numerically accurate integration and also allows for a natural application of
outgoing scattering boundary conditions by adding a complex tail as the last
finite element of the FEM-DVR grid, called exterior complex scaling (ECS). This
project was created for a graduate course in time-dependent quantum
mechanics at UC Davis.

Click on the read hyperlink to find out more!}

%------------------------------------------------


\vspace{1em} % Extra space between major sections


%----------------------------------------------------------------------------------------
%	EDUCATION
%----------------------------------------------------------------------------------------

\spacedlowsmallcaps{Education}\vspace{0.5em}

% \titlerule \vspace{1em}
\hrule \vspace{1em}

\NewEntry{2015-August 2021}{The University of California, Davis}

\Description{\MarginText{Doctor of Philosophy}GPA:\@ $3.9$\ \ $\cdotp$\ \ School: Chemistry\newline
Description: This degree is a PhD in Theoretical Chemical Physics.\newline
Advisors: Prof.~Clyde W. \textsc{McCurdy}, Prof.~Robert. \textsc{Lucchese} (LBNL)}

\NewEntry{Fall 2019}{The University of California, Berkeley}

\Description{\MarginText{Notable Course}CS$294-73$ \textit{Software Engineering
  for Scientific Computing} \newline
School: Computer Science\newline
Grade: A+\newline
Description: This graduate course focused on the seven motifs in scientific computing:
dense and sparse linear algebra, structured and unstructured grid methods,
particle methods, fast Fourier transforms (FFT), and Monte Carlo.\newline
Professor: Phillip \textsc{Colella}\ \ $\cdotp$\ \ \href{mailto:colella@eecs.berkeley.edu}{colella@eecs.berkeley.edu}}

\NewEntry{Spring 2020}{The University of California, Berkeley}

\Description{\MarginText{Notable Course}CS$267$ \textit{Applications of Parallel
  Computers} \newline
School: Computer Science\newline
Grade: A+\newline
Description: Graduate course focused on models for parallel programing. Overview of parallelism on
scientific applications and study of parallel algorithms for linear algebra,
particles, meshes, sorting, FFT, graphs, machine learning, etc. Survey of
parallel machines and machine structures. Programming shared- and
distributed-memory parallel computers, GPUs, and cloud platforms. Parallel
programming languages, compilers, libraries and toolboxes. Data partitioning
techniques. Techniques for synchronization and load balancing. Detailed study
and algorithm/program development of medium sized applications.\newline
Professor: Katherine A. \textsc{Yelick}\ \ $\cdotp$\ \
\href{mailto:yelick@cs.berkeley.edu}{yelick@cs.berkeley.edu}\newline
Professor: James \textsc{Demmel}\ \ $\cdotp$\ \
\href{mailto:demmel@cs.berkeley.edu}{demmel@cs.berkeley.edu}\newline
Professor: Aydin \textsc{Bulu\c{c}}\ \ $\cdotp$\ \
\href{mailto:aydin@eecs.berkeley.edu}{aydin@eecs.berkeley.edu}}

\NewEntry{2007--2009, 2011--2015}{The University of Louisiana, Monroe}

\Description{\MarginText{Bachelor of Science}GPA:\@ $3.46$\ \ $\cdotp$\ \ School: School of Sciences\newline
Major (Concentration): Biology (Chemical Biology)\newline
Personal Courses: Attended formal lectures in Statistical Mechanics, Quantum Mechanics, Electricity and Magnetism, and Relativistic Electrodynamics.\newline
Advisor: Prof.~Gary \textsc{Findley} \& Prof.~Ann \textsc{Findley}}

%------------------------------------------------

\vspace{7em} % Extra space between major sections

%---------------------------------------------------------------------------------------
%TEACHING
%---------------------------------------------------------------------------------------

\spacedlowsmallcaps{Teaching}\vspace{0.5em}

% \titlerule \vspace{1em}
\hrule \vspace{1em}

\NewEntry{Spring 2020}{University of California, Davis}

\Description{\MarginText{Teaching Assistant}Time-Dependent Quantum Mechanics: The first part of this graduate course covers the basic concepts and techniques for solving the time-dependent Schrödinger equation. The initial portion explores the concepts of quantum superpositions, Gaussian wave packets for free and interacting particles, time propagation, the Schrödinger, interaction and Heisenberg representations, time-dependent density matrices, the Wigner phase space distribution, Ehrenfest's theorem, the connection between quantum and classical mechanics in the context of molecular dynamics, the semiclassical wave packet approximation, and time-dependent perturbation theory. The second part of the course turned to applications. Those included absorption and emission of electromagnetic radiation, correlation functions and spectra, molecular dynamics, potential energy surfaces, conical intersections, nonadiabatic transitions and variational transition state theory.}

\NewEntry{Winter 2020}{University of California, Davis}

\Description{\MarginText{Teaching Assistant}Quantum Chemistry: a graduate level discussion of the principles of quantum mechanics and its application to (primarily) stationary state problems in atoms and molecules, including Hartree-Fock calculations of their electronic structure. Using the Psi4 quantum chemistry codes and the Python programming language we performed calculations on small molecules using restricted Hartree-Fock, unrestricted Hartree-Fock, Møller-Plesset perturbation theory (MP2), and configuration interaction (CI) and coupled cluster (CCSD) methods..}

\NewEntry{2015--2016}{University of California, Davis}

\Description{\MarginText{Teaching Assistant}Taught freshman chemistry for two
quarters.  My third quarter I taught quantum mechanics for physical chemistry
students. This course laid the foundation for quantum mechanics needed later in
spectroscopy courses.}

\NewEntry{Spring 2015}{Queens College}

\Description{\MarginText{Teaching Assistant}Taught second semester of freshman chemistry and the corresponding lab.  Created lab and recitation quizzes and was the sole arbiter as to how the courses were conducted.\newline
Assisted Professor: Prof.~Cherice \textsc{Evans}}
%------------------------------------------------

%----------------------------------------------------------------------------------------
%	TALKS AND POSTER PRESENTATIONS
%----------------------------------------------------------------------------------------

\spacedlowsmallcaps{Talks and Posters Presented at Confrences}\vspace{0.5em}

% \titlerule \vspace{1em}
\hrule \vspace{1em}

\Description{\MarginText{2013 SRC Users Meeting}Zachary Streeter, Kamil Krynski, C. M. Evans, and G. L. Findley, ``\textit{Quasi-Free electron in near critical point hydrogen and deuterium,}'' $2013$ SRC Users Meeting, University of Wisconsin Synchrotron Radiation Center, Stoughton, WI, September $27 - 28$, $2013$.}

\Description{\MarginText{2013 SRC Users Meeting}Kamil Krynski, Zachary Streeter, C. M. Evans, and G. L. Findley, ``\textit{Field ionization and photoionization of CH$_3$I perturbed by diatomic molecules: electron scattering in H$_2$, HD, D$_2$, O$_2$ and CO,}'' $2013$ SRC Users Meeting,
University of Wisconsin Synchrotron Radiation Center, Stoughton, WI, September $27 - 28$, $2013$.}

\Description{\MarginText{2014 DAMOP}Cherice Evans, Kamil Krynski, Zachary
Streeter, and G. L. Findley, ``\textit{Field Ionization and Photoionization of
CH$_3$I Perturbed by Diatomic Molecules: Electron Scattering in H$_2$, D$_2$,
O$_2$, and CO,}'' $45$$^{th}$ Annual Meeting of the APS Division of Atomic,
Molecular, and Optical Physics, Madison, WI, June $2--6$, $2014$.}

\Description{\MarginText{2014 DAMOP}Zachary Streeter, Kamil Krynski, C. M.
Evans, and G. L. Findley, ``\textit{The energy of the quasi-free electron in
near critical point H$_2$, D$_2$, and O$_2$,}'' $45$$^{th}$ Annual Meeting of
the APS Division of Atomic, Molecular, and Optical Physics, Madison, WI, June
$2--6$, $2014$.}

\Description{\MarginText{2016 APS}Kamil Krynski, Zachary Streeter, C. M. Evans, and G. L. Findley, ``\textit{Energy of the Quasi-Free Electron in H$_2$, D$_2$, and O$_2$: Probing Intermolecular Potentials within the Local Wigner-Seitz Model,}'' American Physical Society March Meeting, Baltimore, MD, March $14-18$, $2016$.}

\Description{\MarginText{2017 DAMOP}Zachary Streeter, Frank Yip, Dylan P. Reedy, Allen Landers, C. William McCurdy,
``\textit{Classical trajectory studies on the dynamics of one-photon double
photionization of H$_2$O,}`` $48$$^{th}$ Annual Meeting of the APS Division of
Atomic, Molecular, and Optical Physics , Sacramento, CA, June $5--9$, $2017$.}

\Description{\MarginText{2018 ACS}Cherice M. Evans, Jennifer Hare, Baxter Flor,
Kamil Krynski, Zachary Streeter, and G. L. Findley, ``\textit{Energy of the
Quasi-Free Electron in CO and HD:\@ Extension of the Local Wigner-Seitz Model to
Polar Fluids,}`` $225$$^{th}$ ACS National Meeting and Exposition, New Orleans,
LA, March $18--22$, $2018$.}

\Description{\MarginText{2019 DAMOP} Z. L. Streeter, and C. W. McCurdy,
``\textit{Sequential dissociation of H$_2$O$^{++}$ following double
photoionization}`` $50$$^{th}$ Annual Meeting of the APS Division of Atomic,
Molecular, and Optical Physics, Milwaukee, WI, May $27--31$, $2019$.}

\vspace{1em} % Extra space between major sections

%----------------------------------------------------------------------------------------
%	PUBLICATIONS
%----------------------------------------------------------------------------------------

\spacedlowsmallcaps{Publications}\vspace{0.5em}

% \titlerule \vspace{1em}
\hrule \vspace{1em}

\Description{\MarginText{Published}C. M. Evans, Kamil Krynski, Zachary Streeter,
and G. L. Findley, ``\textit{Energy of the Quasi-free Electron in H$_2$, D$_2$
and O$_2$: Probing Intermolecular Potentials within the Local Wigner-Seitz
Model,}'' J. Chem. Phys. $\mathbf{143}$, $224303$ ($2015$)``}

\Description{\MarginText{Published}C. M. Evans, Baxter Flor, Kamil Krynski,
Zachary Streeter, and G. L. Findley,``\textit{Energy of the Quasi-Free Electron
in CO and HD:\@ Probing Intermolecular Potentials within the Local Wigner-Seitz model,}'' J. Chem. Phys. $\mathbf{149}$, $064307$ ($2018$).}

\Description{\MarginText{Published}Zachary L. Streeter, Frank L. Yip, Robert R.
Lucchese, Benoit Gervais, and C. William McCurdy,``\textit{Dissociation dynamics
of the water dication following one-photon double ionization I:\@ Theory,}'' Phys. Rev. A, $\mathbf{98}$, $053429$ ($2018$).}

\Description{\MarginText{Published}D. Reedy, J. B. Williams, B. Gaire, A.
Gatton, M. Weller, A. Menssen, T. Bauer, K. Henrichs, Ph. Burzynski, B. Berry,
Z. L. Streeter, J. Sartor, I. Ben-Itzhak, T. Jahnke, R. D\"orner, Th. Weber, and
A. L. Landers,``\textit{Dissociation dynamics of the water dication following
one-photon double ionization I:\@ Experiment,}'' Phys. Rev. A, $\mathbf{98}$, $053430$ ($2018$).}

\Description{\MarginText{Published}Kirk A. Larsen, Thomas N. Rescigno, Travis
  Severt, Zachary L. Streeter, Wael Iskandar, Saijoscha Heck, Averell Gatton,
  Elio G. Champenois, Richard Strom, Bethany Jochim, Dylan Reedy, Demitri Call,
  Robert Moshammer, Reinhard Dörner, Allen L. Landers, Joshua B. Williams, C.
  William McCurdy, Robert R. Lucchese, Itzik Ben-Itzhak, Daniel S. Slaughter,
  Thorsten Weber,``\textit{Photoelectron and fragmentation dynamics of the H$^+ +$
  H$^+$
dissociative channel in NH$_3$ following direct single-photon double
ionization},`Phys. Rev. Res., $\mathbf{2}$, $043056$ ($2020$).}

\Description{\MarginText{Published}Kirk A. Larsen, Thomas N. Rescigno, Zachary
  L. Streeter, Wael Iskandar, Saijoscha Heck, Averell Gatton, Elio G.
  Champenois, Travis Severt, Richard Strom, Bethany Jochim, Dylan Reedy, Demitri
  Call, Robert Moshammer, Reinhard Dörner, Allen L. Landers, Joshua B. Williams,
  C. William McCurdy, Robert R. Lucchese, Itzik Ben-Itzhak, Daniel S. Slaughter,
  Thorsten Weber,``\textit{Photoionization and dissociation dynamics of the NH+2
  + H+ and NH+ + H+ + H fragmentation channels upon single-photon double ionization
  of NH3 at 61.5eV}''Journal of Physics B., $\mathbf{53}$, $24$ ($2020$).}

\Description{\MarginText{Published} Chuan Cheng, Zachary L. Streeter, Andrew J. Howard 
  Michael Spanner, Robert R. Lucchese, C. William McCurdy, Thomas Weinacht, Phillip H. 
  Buchsbaum, Ruaridh Forbes, ``\textit{Strong-field ionization of water. II. Electronic and nuclear
   dynamics en route to double ionization}'' Physical Review A, $\mathbf{104}$, 
  $02$ ($2021$).}

\Description{\MarginText{Published} Andrew J. Howard, Mathew Britton, Zachary L. Streeter,
  Chuan CHeng, Ruaridh Forbes, Joshua L. Reynolds, Felix Allum, Gregory A. McCracken,
  Ian Gabalski, Robert R. Lucchese, C. William McCurdy, Thomas Weinacht, Phillip H. 
  Bucksbaum, ``\textit{Filming enhanced ionizatin in an 
  ultrafast triatomic slingshot}'' Nature Communications Chemistry, $\mathbf{6}$, 
  $81$ ($2023$).}

\Description{\MarginText{Published} Roger Y. Bello, Frank L. Yip, Zachary Streeter,
  Robert Lucchese, C. William McCurdy, ``\textit{An orbital Basis Set for Double
  Photioionization of Atoms and Molecules}'' Journal of Chemical Theory and 
  Computation, $\mathbf{20}$, $20$ ($2024$).}

\vspace{1em} % Extra space between major sections

%---------------------------------------------------------------------------------------
% Languages
%---------------------------------------------------------------------------------------


%----------------------------------------------------------------------------------------
%	OTHER INFORMATION
%----------------------------------------------------------------------------------------

% \spacedlowsmallcaps{Other Information}\vspace{1em}

% \Description{\MarginText{Awards}2011\ \ $\cdotp$\ \ School of Business Postgraduate Scholarship}

% \vspace{-0.5em} % Negative vertical space to counteract the vertical space between every \Description command

% \Description{2010\ \ $\cdotp$\ \ Top Achiever Award -- Commerce}

%------------------------------------------------

%\vspace{1em}

%\Description{\MarginText{Communication Skills}2010\ \ $\cdotp$\ \ Oral Presentation at the California Business Conference}

%\vspace{-0.5em} % Negative vertical space to counteract the vertical space between every \Description command

%\Description{2009\ \ $\cdotp$\ \ Poster at the Annual Business Conference in Oregon}

%------------------------------------------------

%\vspace{1em}

%\newlength{\langbox} % Create a new length for the length of languages to keep them equally spaced
%\settowidth{\langbox}{English} % Length equals the length of "English" - if you have a longer language in your list put it here

%\Description{\MarginText{Languages}\parbox{\langbox}{\textsc{English}}\ \ $\cdotp$\ \ \ Mothertongue}

%\vspace{-0.5em} % Negative vertical space to counteract the vertical space between every \Description command

%\Description{\parbox{\langbox}{\textsc{Spanish}}\ \ $\cdotp$\ \ \ Intermediate (conversationally fluent)}

%\vspace{-0.5em} % Negative vertical space to counteract the vertical space between every \Description command

%\Description{\parbox{\langbox}{\textsc{Dutch}}\ \ $\cdotp$\ \ \ Basic (simple words and phrases only)}

%\vspace{1em} % Negative vertical space to counteract the vertical space between every \Description command

%------------------------------------------------

%\Description{\MarginText{ Research Interests}Piano\ \ $\cdotp$\ \ Cooking\ \ $\cdotp$\ \ Running\ \ $\cdotp$\ \ Chess\ \ $\cdotp$\ \ Dancing}

%----------------------------------------------------------------------------------------

\end{cv}

\end{document}
